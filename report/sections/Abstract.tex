\thispagestyle{plain}
\begin{center}
    \large
    \textbf{Abstract}
\end{center}

In a production environment, monitoring the energy consumption of industrial machines is critical in assessing energy efficiency, productivity, and in ensuring the machine in operation does not result in high operating costs. Collaborating with a client offering metering hardware, data pertaining to the machine's power consumption over a period of $10$ days in a Swiss media and print company is collected and analyzed. Gaussian Processes, a Bayesian non-parametric learning algorithm, is used to model the underlying physical process of the machines. Training a Gaussian Process model on a period of time that represents the ``best operating conditions" and or before any energy intervention process or new component is applied is called the ``energy baseline model". Using this model, the next day ahead energy consumption is predicted and compared against the actual values. Using statistical process control methodology and charts, the posterior predictive distribution is used to monitor the degree of severity and uncertainty in instantaneous and cumulative changes in machine energy consumption. An example using the model and process control methodology is given for the paper disposal machine.