\thispagestyle{plain}
\begin{center}
    \large
    \textbf{Abstract}
\end{center}

In a production environment, monitoring the energy consumption of industrial machines is critical to assess the energy efficiency, productivity, and to ensure that the machine in operation does not result in high operating costs. Collaborating with an industrial partner offering metering hardware, power consumption data over a period of $10$ days in a Swiss media and print company is collected and analyzed. Gaussian Processes, a Bayesian non-parametric learning algorithm, is used to model a time series of the energy consumption of the machines. Training a Gaussian Process model on a data set that represents the ``best operating conditions" and or before any energy intervention process is applied yields an ``energy baseline model". Using this model, the next day energy consumption is predicted and compared against the actual values. Using statistical process control methodology, the posterior predictive distribution is used to monitor the degree of severity and uncertainty in instantaneous and cumulative changes in machine energy consumption. An example of this approach is presented for a paper disposal machine. Furthermore, to prepare the energy baseline model for deployment on the industrial partner's infrastructure, a Docker container is created encapsulating the model's parameters and inference phase. 