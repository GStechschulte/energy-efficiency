This appendix contains the full list of plausible research questions identified in the preliminary study (the study leading up to the main thesis, i.e., this report).

\begin{enumerate}

    \item Smart energy meters only provide information in regard to inputs used by a piece of equipment and without access to ``output" data from the sensors, how can a baseline EE or benchmark be estimated from energy data alone? Furthermore, how frequently should this benchmark be updated?
    
    \item Could additional data be added or engineered to enhance the solution to research question one such as production data, financial information, temporal aggregations, maintenance logs or weather data?
    
    \item What architecture and tools are best suited for deploying the EEE and PDD models on streaming data?
    
    \item What cloud services and or open-source libraries are needed to productively and cost-effectively deploy the developed EEE and PDD model into production? 
    
    \item What regularization techniques are most effective for the EEE and PDD models in ensuring model performance on new, unseen data?
    
    \item Considering scalability and cost, what is the performance of the model deployment work flows and the cost factors associated with model deployment?
    
    \item Upon addressing research question one, can PDD methods, on top of EEE, be implemented to ensure benefits to that particular piece of equipment in the production process life cycle?
    
    \item What information from the results of the algorithms, hypothetically, needs to be transmitted back to the “production manager” in addressing questions one and seven so he/she can make informed production actions?
    
\end{enumerate}